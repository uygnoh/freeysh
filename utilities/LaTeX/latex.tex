\documentclass[UTF8]{article}
\usepackage{ctex}
\begin{document}
	% 罗马字体,无衬线字体,打字机字体设置
	\textrm{罗马字体 Roman Family}\\
	\textsf{无衬线字体 Sans Serif Family}\\
	\texttt{打字机字体 Typewriter Family}\\
	%字体声明,需要大括号确定范围,否则后续不一致
	{\rmfamily 罗马字体 Roman Family}\\
	{\sffamily 无衬线字体 Sans Serif Family}\\
	{\ttfamily 打字机字体 Typewriter Family}
	
	
	%中文字体声明,ctex宏包下
	{\songti 宋体}\\
	{\heiti 黑体}\\
	{\fangsong 仿宋}\\
	{\kaishu 楷书}
	
	
	%字体形状设置:直立,斜体,伪斜体,小型大写
	\textup{Upright Shape 直立}\\
	\textbf{Boldface Shape 粗体}\\
	\textit{Italic Shape 斜体}\\
	\textsl{Slanted Shape 伪斜体}\\
	\textsc{Small Caps Shape 小型大写}
	%字体设置声明
	{\upshape Upright Shape 直立}\\
	{\itshape Italic Shape 斜体}\\
	{\slshape Slanted Shape 伪斜体}\\
	{\scshape Small Caps Shape 小型大写}
	
	
	%字体大小设置
	{\tiny Word 字}\\
	{\scriptsize Word 字}\\
	{\footnotesize Word 字}\\
	{\small Word 字}\\
	{\normalsize Word 字}\\
	{\large Word 字}\\
	{\Large Word 字}\\
	{\LARGE Word 字}\\
	{\huge Word 字}\\
	{\Huge Word 字}
\end{document}







